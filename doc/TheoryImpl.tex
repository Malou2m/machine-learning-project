\documentclass[11pt, oneside]{article}   	
\usepackage{geometry}                	
\geometry{letterpaper}                   	 
\usepackage{graphicx}	

\usepackage{amsmath}
\usepackage{algorithm}
\usepackage[noend]{algpseudocode}			
\makeatletter
\def\BState{\State\hskip-\ALG@thistlm}
\renewcommand{\algorithmicforall}{\textbf{for each}}
\makeatother					
									
\usepackage{amssymb}
\usepackage[utf8]{inputenc}
\usepackage[T1]{fontenc}
\usepackage{float}
\restylefloat{table}
\restylefloat{algorithm}


\title{Theoriefragen zur Implementationsaufgabe}
\author{K\"upper Joel,
Ockenfels Malou,
Schulz Daniel}

\begin{document}
\maketitle

\section{5.3) Normalization and Tie-Breaking}
\subsection{a) Entscheidungsfunktion für getWinner}
Problem: \textit{Es kann vorkommen, dass mehr als eine Klasse am meisten Stimmen von den Nachbarn bekommen hat. Welche Klasse soll dann ausgewählt werden?}\newline
Die Idee ist, dass falls mehr als eine Klasse am meisten Stimmen bekommen hat, jene klasse ausgewählt wird, die generell im Data-set am meisten vorkommt. 
Somit ist die Wahrscheinlichkeit, richtig zu liegen, am höchsten.
Den Fall, dass die beiden \emph{gewinnenden} Klassen auch gleichzeitig gleich häufig im Data-set vorkommen wird hier in der Implementierung nicht beachtet, da die Wahrscheinlichkeit doch sehr gering ist, vorallem bei den vorliegenden Data-sets.\newline
Für diesen Fall könnte jedoch so vorgegangen werden, dass jene Klasse vorhergesagt wird, die am nächsten (k=1, =2, ...) am Beispiel liegt.
Dann gibt es aber Probleme bei zu hohem Rauschen in den Data-sets.\newline
Eine weitere Möglichkeit wäre, dass man k+=1 inkrementiert und so einen weiteren Nachbarn (oder zwei, drei, ...) hinzuzieht, um eine eindeutigere Mehrheit zu finden. 
Dabei tritt aber das Problem der zu generrellen Vorhersage auf. 
Da in der Aufgabe keine Angabe über jene Priorisierung gemacht wurde, werden die Denkansätze hier nur erähnt und nicht implementiert.\newline
Der folgende Pseudocode illustriert, wie das Zählen umgesetzt wurde.

\begin{algorithm}[H]
\caption{getWinner Entscheidungsfunktion}\label{euclid}
\begin{algorithmic}[1]
\Function{getWinner}{data, votes}
\State $max \gets votes.getHighestVote()$
\State $classesWithMaxVotes \gets null$
\ForAll{$vote \in votes$}
\If{vote.numberOfVotes() == max}
	\State $classesWithMaxVotes.add(vote);$
\EndIf
\EndFor
\If{classesWithMaxVotes.size == 1}
	\State $return \gets classesWithMaxVotes.first();$
\Else
	\State $counter.init(0);$
	\ForAll{$instance \in data$}
		\If{classesWithMaxVotes.contains(instance.getClass())}
			\State $counter.incrementClass(instance.getClass());$
		\EndIf
	\EndFor
\EndIf
\State $return \gets counter.getBiggest().getClass();$

\EndFunction
\end{algorithmic}
\end{algorithm}
\pagebreak

\subsection{c) Gleicher Abstand mehrerer Instanzen}
Problem: \textit{Es kann vorkommen, dass mehrere Instanzen den glichen Abstand vom Beispiel haben. Wie soll dabei vorgegangen werden?}\newline
Das ist prinzipiell nur ein Problem, wenn der k'te Nachbar den gleichen Abstand hat wie k+1, k+2, ... 
Wenn die Nachbarn eins und zwei bspw. bei k=4 den gleichen Abstand haben, gibt es kein Problem.\newline
Nun, hier ist die Idee, wie oben schon beschrieben, aber nun unausweichlich, k zu inkrementieren und alle Nachbarn, die den gleichen Abstand haben wie der k'te Nachbar auch noch als nearst Neighbor zu bezeichnen und entsprechend zu berücksichtigen.\newline
Auch hier der Preudocode, wie es umgesetzt wurde:

\begin{algorithm}[H]
\caption{getNearest Entscheidungsfunktion}\label{euclid}
\begin{algorithmic}[1]
\Function{getNearest}{sortedListOfNeighbors, k}

\State $sortedNeighborsByDistance \gets null;$
\ForAll{$neighbor \in sortedListOfNeighbors$}
	\State $sortedNeighborsByDistance.addAt(neighbor.getDistance(), neighbor);$
\EndFor
\State $nearests \gets null;$
\ForAll{$setOfNeighborsWithSameDistance \in sortedNeighborsByDistance$}
	\State $nearests.addAll(setOfNeighborsWithSameDistance);$
	\If{$nearest.size() >= k$}
		\State break;
	\EndIf
\EndFor
\State $return \gets nearests;$

\EndFunction
\end{algorithmic}
\end{algorithm}

\end{document}

